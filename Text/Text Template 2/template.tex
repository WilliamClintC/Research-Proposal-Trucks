% Options for packages loaded elsewhere
\PassOptionsToPackage{unicode}{hyperref}
\PassOptionsToPackage{hyphens}{url}
\PassOptionsToPackage{dvipsnames,svgnames,x11names}{xcolor}
%
\documentclass[
  8pt,
  12pt]{article}

\usepackage{amsmath,amssymb}
\usepackage{iftex}
\ifPDFTeX
  \usepackage[T1]{fontenc}
  \usepackage[utf8]{inputenc}
  \usepackage{textcomp} % provide euro and other symbols
\else % if luatex or xetex
  \usepackage{unicode-math}
  \defaultfontfeatures{Scale=MatchLowercase}
  \defaultfontfeatures[\rmfamily]{Ligatures=TeX,Scale=1}
\fi
\usepackage{lmodern}
\ifPDFTeX\else  
    % xetex/luatex font selection
\fi
% Use upquote if available, for straight quotes in verbatim environments
\IfFileExists{upquote.sty}{\usepackage{upquote}}{}
\IfFileExists{microtype.sty}{% use microtype if available
  \usepackage[]{microtype}
  \UseMicrotypeSet[protrusion]{basicmath} % disable protrusion for tt fonts
}{}
\makeatletter
\@ifundefined{KOMAClassName}{% if non-KOMA class
  \IfFileExists{parskip.sty}{%
    \usepackage{parskip}
  }{% else
    \setlength{\parindent}{0pt}
    \setlength{\parskip}{6pt plus 2pt minus 1pt}}
}{% if KOMA class
  \KOMAoptions{parskip=half}}
\makeatother
\usepackage{xcolor}
\setlength{\emergencystretch}{3em} % prevent overfull lines
\setcounter{secnumdepth}{5}
% Make \paragraph and \subparagraph free-standing
\makeatletter
\ifx\paragraph\undefined\else
  \let\oldparagraph\paragraph
  \renewcommand{\paragraph}{
    \@ifstar
      \xxxParagraphStar
      \xxxParagraphNoStar
  }
  \newcommand{\xxxParagraphStar}[1]{\oldparagraph*{#1}\mbox{}}
  \newcommand{\xxxParagraphNoStar}[1]{\oldparagraph{#1}\mbox{}}
\fi
\ifx\subparagraph\undefined\else
  \let\oldsubparagraph\subparagraph
  \renewcommand{\subparagraph}{
    \@ifstar
      \xxxSubParagraphStar
      \xxxSubParagraphNoStar
  }
  \newcommand{\xxxSubParagraphStar}[1]{\oldsubparagraph*{#1}\mbox{}}
  \newcommand{\xxxSubParagraphNoStar}[1]{\oldsubparagraph{#1}\mbox{}}
\fi
\makeatother


\providecommand{\tightlist}{%
  \setlength{\itemsep}{0pt}\setlength{\parskip}{0pt}}\usepackage{longtable,booktabs,array}
\usepackage{calc} % for calculating minipage widths
% Correct order of tables after \paragraph or \subparagraph
\usepackage{etoolbox}
\makeatletter
\patchcmd\longtable{\par}{\if@noskipsec\mbox{}\fi\par}{}{}
\makeatother
% Allow footnotes in longtable head/foot
\IfFileExists{footnotehyper.sty}{\usepackage{footnotehyper}}{\usepackage{footnote}}
\makesavenoteenv{longtable}
\usepackage{graphicx}
\makeatletter
\def\maxwidth{\ifdim\Gin@nat@width>\linewidth\linewidth\else\Gin@nat@width\fi}
\def\maxheight{\ifdim\Gin@nat@height>\textheight\textheight\else\Gin@nat@height\fi}
\makeatother
% Scale images if necessary, so that they will not overflow the page
% margins by default, and it is still possible to overwrite the defaults
% using explicit options in \includegraphics[width, height, ...]{}
\setkeys{Gin}{width=\maxwidth,height=\maxheight,keepaspectratio}
% Set default figure placement to htbp
\makeatletter
\def\fps@figure{htbp}
\makeatother

\addtolength{\oddsidemargin}{-.5in}%
\addtolength{\evensidemargin}{-1in}%
\addtolength{\textwidth}{1in}%
\addtolength{\textheight}{1.7in}%
\addtolength{\topmargin}{-1in}%
\makeatletter
\@ifpackageloaded{caption}{}{\usepackage{caption}}
\AtBeginDocument{%
\ifdefined\contentsname
  \renewcommand*\contentsname{Table of contents}
\else
  \newcommand\contentsname{Table of contents}
\fi
\ifdefined\listfigurename
  \renewcommand*\listfigurename{List of Figures}
\else
  \newcommand\listfigurename{List of Figures}
\fi
\ifdefined\listtablename
  \renewcommand*\listtablename{List of Tables}
\else
  \newcommand\listtablename{List of Tables}
\fi
\ifdefined\figurename
  \renewcommand*\figurename{Figure}
\else
  \newcommand\figurename{Figure}
\fi
\ifdefined\tablename
  \renewcommand*\tablename{Table}
\else
  \newcommand\tablename{Table}
\fi
}
\@ifpackageloaded{float}{}{\usepackage{float}}
\floatstyle{ruled}
\@ifundefined{c@chapter}{\newfloat{codelisting}{h}{lop}}{\newfloat{codelisting}{h}{lop}[chapter]}
\floatname{codelisting}{Listing}
\newcommand*\listoflistings{\listof{codelisting}{List of Listings}}
\makeatother
\makeatletter
\makeatother
\makeatletter
\@ifpackageloaded{caption}{}{\usepackage{caption}}
\@ifpackageloaded{subcaption}{}{\usepackage{subcaption}}
\makeatother
\ifLuaTeX
  \usepackage{selnolig}  % disable illegal ligatures
\fi
\usepackage[]{natbib}
\bibliographystyle{agsm}
\usepackage{bookmark}

\IfFileExists{xurl.sty}{\usepackage{xurl}}{} % add URL line breaks if available
\urlstyle{same} % disable monospaced font for URLs
\hypersetup{
  pdftitle={Zoning: A Barrier or Solution to Truck Parking Infrastructure Shortages?},
  pdfauthor={William Co},
  colorlinks=true,
  linkcolor={blue},
  filecolor={Maroon},
  citecolor={Blue},
  urlcolor={Blue},
  pdfcreator={LaTeX via pandoc}}


\begin{document}


\def\spacingset#1{\renewcommand{\baselinestretch}%
{#1}\small\normalsize} \spacingset{1}


%%%%%%%%%%%%%%%%%%%%%%%%%%%%%%%%%%%%%%%%%%%%%%%%%%%%%%%%%%%%%%%%%%%%%%%%%%%%%%

\date{March 9, 2025}
\title{\bf Zoning: A Barrier or Solution to Truck Parking Infrastructure
Shortages?}
\author{
William Co\thanks{true}\\
Department of Economics, University of British Columbia\\
}
\maketitle

\bigskip
\bigskip
\begin{abstract}

\end{abstract}


\newpage
\spacingset{1.9} % DON'T change the spacing!

\section{Institutional Setting}\label{institutional-setting}

The Federal Motor Carrier Safety Administration (FMCSA), maintains a
comprehensive database known as the Crash File. This dataset records all
reported motor vehicle crashes in the United States, providing detailed
insights into the nature and conditions of each accident. Key attributes
include the type of vehicle involved (e.g., trucks, motorcycles, or
buses), the circumstances of the crash (e.g., involving a parked
vehicle), the number of vehicles involved, any fatalities or injuries,
and relevant weather conditions, and observations. Our data is
(1990-Present) FMCA Crash file from USDOT
(\phantomsection\label{sec:appendix-a}\hyperref[sec-a.-visualization-of-dataset.-]{Appendix
A}).

Our dataset spans from 1990 to the present, with a focus on records from
1993 to 2016. A unique feature of this dataset is its ability to
distinguish trucks as a specific variable, allowing for a granular
analysis of truck-involved collisions. Furthermore, it offers detailed
information on accident circumstances, such as whether a truck was
illegally parked, a distinction not commonly found in other datasets.

We also incorporate a digitized data set tracking truck stop creation
from 2006 to 2016. This data set enables us to analyze the impact of new
truck stops on accident patterns at the county level. Notably, this
period lacks significant policy reforms or major events that could
confound our analysis.

Additionally, we utilize the Wharton Land Use Regulation Index (WLIURA)
dataset by \citet{gyourkoNewMeasureLocal2008}, which measures the zoning
restrictiveness of various locations. This dataset allows us to examine
the influence of local zoning laws on crash patterns. We further refine
our analysis by incorporating zoning classifications from
\citet{puentesTraditionalReformedReview2006}, which categorize land use
regulations into four distinct zoning types. These classifications serve
as control variables in our study.

\section{Event Study Model}\label{event-study-model}

The change in the number of truck stops from 2006 to 2016 is modeled as:

\[
\Delta \text{NumTruckStop}_{t=[2006-2016]} = 
\sum_{i=1993}^{2006} \beta_{t=i} \text{Accident}_{t=i}\cdot \text{Fatalities}_{t=i} + \gamma_{t} X_{t} + \epsilon_{t}
\]

where year \(t\) represents time \(i\),
\(\Delta\text{NumTruckStop}_{t}\) denotes the change in truck stops,
\(\text{Accident}_{i}\) is an event dummy indicating the presence of an
accident, and \(\text{Fatalities}_{i}\) represents associated
fatalities. \(X_{t}\) consists of control variables, including zoning
categories: Traditional (zoning unchanging), Exclusion (zoning
difficult), Reform (zoning friendly), and Wild Wild Texas (no zoning).
Finally, \(\epsilon_{t}\) represents the error term for year \(t\). We
also add 2 dummy controls for counties with high and low zoning
restriction. More possible controls would include,region, county budget,
county weather patterns and population. In this model, we assume that
the occurrence of accidents and their severity are uncorrelated with the
error term, ensuring that any estimated effect of these variables on the
change in truck stops is not biased by omitted factors. In addition, the
model includes several zoning measures specifically the categories
Traditional, Exclusion, Reform, and Wild Wild Texas, as well as dummy
controls for high and low zoning restrictions. It is assumed that these
zoning variables are not perfectly collinear, allowing for valid
coefficient estimation and clear identification of each variable's
effect. Furthermore, the model is based on the assumption of
unidirectional causality, meaning that changes in the number of truck
stops do not lead to an increase in accidents. This assumption is
critical to ensure that the estimated relationships truly reflect the
impact of accidents and severity on truck stop changes, rather than the
reverse.

The identification assumption is that in the absence of designated truck
parking, trucks will resort to illegal parking, leading to observable
accidents. In response, counties, driven by public safety concerns and
the goal of reducing accidents, are incentive to increase truck parking
availability. This, in turn, results in the creation of additional truck
stops. The causal mechanism can be summarized as follows:

\emph{Insufficient Truck Parking → Illegal Truck Parking → Observed
Accidents → Increased Demand for Truck Parking → Observed Increase in
Truck Parking Capacity.}

We can test this assumption by observing the coefficients surrounding,
each zoning category and high/low restriction coefficient. We expect to
see Traditional (zoning unchanging) with a insignificant coefficient,
Exclusion (zoning difficult) with a negative coefficient, Reform (zoning
friendly) with a positive coefficient, and Wild Wild Texas (no zoning)
with a positive coefficient. High restriction places will have

Potential sources of bias would be overestimation of accidents. It is
really difficult for a truck to legally park anywhere. It could be well
the case that the truck did every reasonable measure to park properly
but would still get into a accident. This would overestimate accidents
and underestimate our coefficient. Another source of bias would be
underestimation of severity of accidents. Counties have a incentive to
maintain a safe public image, which would include minimizing severity of
accidents, underestimating our coefficient estimate. Furthermore, zoning
classifications literature, support the idea that time variation of
counties overtime follow their zoning categories. For example, Exclusion
(zoning difficult) would become more difficult over time and vice
versa\citep{mclaughlinLandUseRegulation2012}. This would overestimate
our coefficient estimates as t increases.

We will cluster standard errors at the region/state level because
different regions/state exhibit distinct accident and trucking profiles.
Some regions/states may have a more prominent trucking culture than
others, leading to systematic differences in accident rates and truck
stop availability. Additionally, regional weather conditions may
influence both the frequency of accidents and the feasibility of truck
stop construction.

By clustering at the region/state level, we assume that observations
within each region/state are correlated, while observations across
different states remain independent. This accounts for within-state
dependencies, such as shared infrastructure, regulations, and economic
conditions. Clustering at this level also ensures a sufficient number of
clusters, which is crucial for obtaining reliable standard error
estimates.


  \bibliography{bibliography.bib}


\end{document}
