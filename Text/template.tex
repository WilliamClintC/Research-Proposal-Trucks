% Options for packages loaded elsewhere
\PassOptionsToPackage{unicode}{hyperref}
\PassOptionsToPackage{hyphens}{url}
\PassOptionsToPackage{dvipsnames,svgnames,x11names}{xcolor}
%
\documentclass[
  12pt]{article}

\usepackage{amsmath,amssymb}
\usepackage{iftex}
\ifPDFTeX
  \usepackage[T1]{fontenc}
  \usepackage[utf8]{inputenc}
  \usepackage{textcomp} % provide euro and other symbols
\else % if luatex or xetex
  \usepackage{unicode-math}
  \defaultfontfeatures{Scale=MatchLowercase}
  \defaultfontfeatures[\rmfamily]{Ligatures=TeX,Scale=1}
\fi
\usepackage{lmodern}
\ifPDFTeX\else  
    % xetex/luatex font selection
\fi
% Use upquote if available, for straight quotes in verbatim environments
\IfFileExists{upquote.sty}{\usepackage{upquote}}{}
\IfFileExists{microtype.sty}{% use microtype if available
  \usepackage[]{microtype}
  \UseMicrotypeSet[protrusion]{basicmath} % disable protrusion for tt fonts
}{}
\makeatletter
\@ifundefined{KOMAClassName}{% if non-KOMA class
  \IfFileExists{parskip.sty}{%
    \usepackage{parskip}
  }{% else
    \setlength{\parindent}{0pt}
    \setlength{\parskip}{6pt plus 2pt minus 1pt}}
}{% if KOMA class
  \KOMAoptions{parskip=half}}
\makeatother
\usepackage{xcolor}
\setlength{\emergencystretch}{3em} % prevent overfull lines
\setcounter{secnumdepth}{5}
% Make \paragraph and \subparagraph free-standing
\makeatletter
\ifx\paragraph\undefined\else
  \let\oldparagraph\paragraph
  \renewcommand{\paragraph}{
    \@ifstar
      \xxxParagraphStar
      \xxxParagraphNoStar
  }
  \newcommand{\xxxParagraphStar}[1]{\oldparagraph*{#1}\mbox{}}
  \newcommand{\xxxParagraphNoStar}[1]{\oldparagraph{#1}\mbox{}}
\fi
\ifx\subparagraph\undefined\else
  \let\oldsubparagraph\subparagraph
  \renewcommand{\subparagraph}{
    \@ifstar
      \xxxSubParagraphStar
      \xxxSubParagraphNoStar
  }
  \newcommand{\xxxSubParagraphStar}[1]{\oldsubparagraph*{#1}\mbox{}}
  \newcommand{\xxxSubParagraphNoStar}[1]{\oldsubparagraph{#1}\mbox{}}
\fi
\makeatother


\providecommand{\tightlist}{%
  \setlength{\itemsep}{0pt}\setlength{\parskip}{0pt}}\usepackage{longtable,booktabs,array}
\usepackage{calc} % for calculating minipage widths
% Correct order of tables after \paragraph or \subparagraph
\usepackage{etoolbox}
\makeatletter
\patchcmd\longtable{\par}{\if@noskipsec\mbox{}\fi\par}{}{}
\makeatother
% Allow footnotes in longtable head/foot
\IfFileExists{footnotehyper.sty}{\usepackage{footnotehyper}}{\usepackage{footnote}}
\makesavenoteenv{longtable}
\usepackage{graphicx}
\makeatletter
\def\maxwidth{\ifdim\Gin@nat@width>\linewidth\linewidth\else\Gin@nat@width\fi}
\def\maxheight{\ifdim\Gin@nat@height>\textheight\textheight\else\Gin@nat@height\fi}
\makeatother
% Scale images if necessary, so that they will not overflow the page
% margins by default, and it is still possible to overwrite the defaults
% using explicit options in \includegraphics[width, height, ...]{}
\setkeys{Gin}{width=\maxwidth,height=\maxheight,keepaspectratio}
% Set default figure placement to htbp
\makeatletter
\def\fps@figure{htbp}
\makeatother

\addtolength{\oddsidemargin}{-.5in}%
\addtolength{\evensidemargin}{-1in}%
\addtolength{\textwidth}{1in}%
\addtolength{\textheight}{1.7in}%
\addtolength{\topmargin}{-1in}%
\makeatletter
\@ifpackageloaded{caption}{}{\usepackage{caption}}
\AtBeginDocument{%
\ifdefined\contentsname
  \renewcommand*\contentsname{Table of contents}
\else
  \newcommand\contentsname{Table of contents}
\fi
\ifdefined\listfigurename
  \renewcommand*\listfigurename{List of Figures}
\else
  \newcommand\listfigurename{List of Figures}
\fi
\ifdefined\listtablename
  \renewcommand*\listtablename{List of Tables}
\else
  \newcommand\listtablename{List of Tables}
\fi
\ifdefined\figurename
  \renewcommand*\figurename{Figure}
\else
  \newcommand\figurename{Figure}
\fi
\ifdefined\tablename
  \renewcommand*\tablename{Table}
\else
  \newcommand\tablename{Table}
\fi
}
\@ifpackageloaded{float}{}{\usepackage{float}}
\floatstyle{ruled}
\@ifundefined{c@chapter}{\newfloat{codelisting}{h}{lop}}{\newfloat{codelisting}{h}{lop}[chapter]}
\floatname{codelisting}{Listing}
\newcommand*\listoflistings{\listof{codelisting}{List of Listings}}
\makeatother
\makeatletter
\makeatother
\makeatletter
\@ifpackageloaded{caption}{}{\usepackage{caption}}
\@ifpackageloaded{subcaption}{}{\usepackage{subcaption}}
\makeatother
\ifLuaTeX
  \usepackage{selnolig}  % disable illegal ligatures
\fi
\usepackage[]{natbib}
\bibliographystyle{agsm}
\usepackage{bookmark}

\IfFileExists{xurl.sty}{\usepackage{xurl}}{} % add URL line breaks if available
\urlstyle{same} % disable monospaced font for URLs
\hypersetup{
  pdftitle={Zoning: A Barrier or Solution to Truck Parking Infrastructure Shortages?},
  pdfauthor={William Co},
  colorlinks=true,
  linkcolor={blue},
  filecolor={Maroon},
  citecolor={Blue},
  urlcolor={Blue},
  pdfcreator={LaTeX via pandoc}}


\begin{document}


\def\spacingset#1{\renewcommand{\baselinestretch}%
{#1}\small\normalsize} \spacingset{1}


%%%%%%%%%%%%%%%%%%%%%%%%%%%%%%%%%%%%%%%%%%%%%%%%%%%%%%%%%%%%%%%%%%%%%%%%%%%%%%

\date{January 14, 2025}
\title{\bf Zoning: A Barrier or Solution to Truck Parking Infrastructure
Shortages?}
\author{
William Co\thanks{true}\\
Department of Economics, University of British Columbia\\
}
\maketitle

\bigskip
\bigskip
\begin{abstract}
This study explores the impact of local zoning regulations on the
growing shortage of truck parking in the United States. We utilize
traffic accident data as a proxy for truck parking demand to examine how
zoning restrictions affect parking capacity. Employing an event study
design, we categorize zoning regimes into Traditional, Exclusion,
Reform, and Wild Wild Texas. Additionally, we implement a
difference-in-differences approach to compare high-restrictive and
low-restrictive areas.
\end{abstract}


\newpage
\spacingset{1.9} % DON'T change the spacing!

\section{Agenda}\label{agenda}

\begin{enumerate}
\def\labelenumi{\arabic{enumi}.}
\item
  Quantifying Truck Parking Shortage.~
\item
  Do local economies respond to the need for truck parking or is it
  hindered?
\item
  Does land regulation limit truck parking creation?
\end{enumerate}

\section{Introduction}\label{sec-intro}

talk about illegal to overwork yourself but hereres no parking

zonign welfare

is there a identifieabl undersupply of trucking regulation

truck parkign accdeints effect on truck stop creation

Question: come up with a well-defined research question (i.e., what is
the effect of X on Y?) ◮ Motivation: tell us a convincing argument as to
why economists/policymakers should care about knowing an answer to that
question

Question: come up with a well-defined research question (i.e., what is
the effect of X on Y?) ◮ Motivation: tell us a convincing argument as to
why economists/policymakers should care about knowing an answer to that
question. ◮ Conceptual Framework: what economic model informs us about a
potential answer to the question, and what are the comparative statics
you can derive from the model

Question: come up with a well-defined research question (i.e., what is
the effect of X on Y?) ◮ Motivation: tell us a convincing argument as to
why economists/policymakers should care about knowing an answer to that
question. ◮ Conceptual Framework: what economic model informs us about a
potential answer to the question, and what are the comparative statics
you can derive from the model? ◮ Contribution: If it's such an important
question, why somebody couldn't have done it before? In other words,
what is your unique empirical contribution

The United States hosts 90,056 local governments, each imposing unique
zoning restrictions that shape land use. Among the pressing concerns
influenced by these regulations is the growing shortage of truck parking
\citep{american-trucking-associationNationalTruckParking2023}.
Inadequate truck parking has led to dangerous or illegal practices, such
as parking on highway shoulders or in unauthorized areas, which
heightens traffic accident risks and imposes economic costs like
increased fuel consumption, delivery delays, and inflated goods prices
\citep{usdotJasonsLawTruck2015}.

Despite the significant implications, there is limited empirical
research quantifying the relationship between truck parking demand and
the regulatory environment influencing its availability. Zoning
regulations, in particular, often play a more decisive role in truck
parking accessibility than geographic or transportation network factors
\citep{shertzerZoningEconomicGeography2018}. This study seeks to bridge
this gap by leveraging traffic accident data as a proxy for truck
parking demand and analyzing how land-use regulations impact parking
availability.

Specifically, this research aims to test two hypotheses:

\begin{enumerate}
\def\labelenumi{\arabic{enumi}.}
\item
  If restrictive local zoning laws drive truck parking shortages,
  regions with high truck parking-related accidents should exhibit
  minimal correlation with increased truck stop capacity.
\item
  Parking accidents involving trucks can serve as a quantitative proxy
  for truck stop demand, enabling the evaluation of truck stop shortages
  across regions.
\end{enumerate}

this study employs an event study design, categorizing zoning regimes
into four types---Traditional, Exclusion, Reform, and Wild Wild
Texas---based on \citep{puentesTraditionalReformedReview2006} (See
\phantomsection\label{sec:appendix-b}\hyperref[sec-b.-map-of-zoning-categories]{Appendi}\hyperref[sec:appendix-b]{x
B}) . We hypothesize that restrictive zoning regimes (Traditional and
Exclusion) are less responsive to truck parking demand compared to
flexible regimes (Reform and Wild Wild Texas), resulting in lower truck
parking capacity despite evident needs.

Additionally, this study will employ a difference-in-differences design
to compare high-restrictive zoning areas with low-restrictive zoning
areas. By controlling for the four zoning categories---Traditional,
Exclusion, Reform, and Wild Wild Texas---we analyze the effects of a
major trucking accident as the event of interest. This approach will
allow us to assess how zoning restrictiveness influences truck parking
capacity and whether high-restrictive regimes exacerbate the challenges
associated with truck parking shortages in the aftermath of such
incidents.

\section{Background}\label{background}

Existing studies on zoning provide valuable context but are limited in
scope. Research often focuses on single municipalities
\citep{shertzerRaceEthnicityDiscriminatory2016, glaeserCausesConsequencesLand2009}
or international contexts \citep{anagolEstimatingEconomicValue2021}.
Furthermore, most literature emphasizes residential zoning
\citep{lensStrictLandUse2016, huangResidentialLandUse2012}or office
space \citep{cheshireOfficeSpaceSupply2008}, leaving industrial zoning
and its implications for truck parking largely unexplored.

Initially designed to balance public welfare and economic development,
zoning regulations have evolved, sometimes adapting to market forces or
catering to local stakeholder interests, such as middle-class homeowners
\citep{fischelEconomicHistoryZoning2024}. While zoning has the potential
to enhance economic productivity, it can also introduce inefficiencies,
particularly in industrial applications
\citep{mcdonaldPDFEconomicsZoning2012}. Fragmented zoning governance
often discourages communities from accommodating truck parking, despite
its regional benefits, due to localized decision-making dynamics.
Furthermore, it is unclear whether the current state of land regulation
optimizes welfare \citep{osmanRestrictiveLandUse2020}. This paper aims
to address this gap within the context of truck parking shortages these
challenges,.

This research contributes to the broader discourse on zoning's economic
impact, extending the analysis to the critical issue of truck parking
infrastructure. By examining the interplay between zoning
classifications, parking-related accidents, and truck stop capacity,
this study offers insights for policymakers aiming to mitigate the
externalities of inadequate truck parking through thoughtful zoning
reforms.

\section{Data}\label{data}

Our data is (1990-Present) FMCA (Federal Motor Carrier Safety
Administration) Crash file from USDOT (Department of Transportation)
(\phantomsection\label{sec:appendix-a}\hyperref[sec-a.-visualization-of-dataset.-]{Appendix
A}). We can see type of vehicle (ex. trucks), nature of the crash (ex.
Crashed involving a ``parked'' vehicle), fatalities, injuries, number of
vehicles involved, etc. We will aslo use WLIURA (zoning restriction
index) dataset by \citet{gyourkoNewMeasureLocal2008} and zoning
classifications used by \citet{puentesTraditionalReformedReview2006} .

\section{\texorpdfstring{\textbf{Strategy}}{Strategy}}\label{strategy}

With no truck parking available trucks illegally park causing observed
accidents such that areas increase truck parking availability.

No Truck Parking → Trucks will illegally park → \textbf{Accidents Occur}
(observed)→Truck Parking Demand increases → \textbf{Truck Parking
Capacity Increase} (observation)

\subsection{Event Study Model}\label{event-study-model}

The equation to estimate the effect of the Truck Parking Accident (TPA)
on the creation of truck stops is specified as follows:

\[
Y_{tj} = \sum_{i=-6}^{6} \beta_i \text{TPA}_{i,j} + \gamma X_{tj} + \epsilon_{tj}
\]

Where:

\begin{itemize}
\tightlist
\item
  \(Y_{tj}\) is the change in truck stop capacity in year \(t\) for
  category \(j\).
\item
  \(\beta_i\) are the coefficients for the event dummies
  (\(\text{TPA}_{i,j}\)), where \(i\) represents the relative year to
  the Truck Parking Accident
  (\(i = -6, -5, -4, -3, -2, -1, 0, 1, 2, 3, 4, 5, 6\)).
\item
  \(\text{TPA}_{i,j}\) is the event dummy indicating the presence of a
  Truck Parking Accident in relative year \(i\) for category \(j\).
  Specifically, when \(i = 0\), this corresponds to the year in which
  the Truck Parking Accident occurs.
\item
  \(\gamma\) represents the coefficients for the control variables
  (\(X_{tj}\)).
\item
  \(X_{tj}\) are the control variables in year \(t\) for category \(j\).
\item
  \(\epsilon_{tj}\) is the error term for year \(t\) and category \(j\).
\item
  \(j\) indicates a specific category used to isolate subsets of the
  data.
\end{itemize}

To analyze the impact of the Truck Parking Accident across different
zoning categories, I will estimate to each corresponding category:

\begin{enumerate}
\def\labelenumi{\arabic{enumi}.}
\item
  \textbf{Traditional}: This category evaluates the effects in
  conventional settings with typical zoning regulations.
\item
  \textbf{Exclusion}: This category examines the impacts in areas where
  truck stops are limited or restricted by zoning laws.
\item
  \textbf{Reform}: This category focuses on regions undergoing policy or
  structural reforms related to truck parking.
\item
  \textbf{Wild Wild Texas}: This category investigates the unique
  circumstances and effects in Texas, a state known for its lack of
  zoning regulation.
\end{enumerate}

\subsection{Difference-in-Differences (DiD)
model}\label{difference-in-differences-did-model}

Furthermore, using the WLRUI index data and a DiD design, I will compare
locations with a high restriction index to those with a low restriction
index, controlling for truck parking categories. The model specification
is as follows:

\[
Y_{it} = \sum_{t=-6}^{6} \theta_t HR_{it} + \sum_{t=-6}^{6} \phi_t Post_{it} + \sum_{t=-6}^{6} \psi_t (HR_{it} \times Post_{it}) + \gamma X_{it} + \epsilon_{it}
\]

Where

\begin{itemize}
\item
  \(Y_{it}\) is the chaneg in truck parking capacity for category \(i\)
  at time \(t\).
\item
  \(HR_{it}\) is the high restriction index indicator for category \(i\)
  at time \(t\).
\item
  \(Post_{it}\) is the indicator for the post-TPA period for category
  \(i\) at time \(t\).
\item
  \(HR_{it} \times Post_{it}\) is the interaction term that captures the
  treatment effect of being in a high restriction area after the truck
  parking accident.
\item
  \(\theta_t\) represents unique coefficients for the high restriction
  index across time periods \(t\).
\item
  \(\phi_t\) represents unique coefficients for the post-TPA period
  across time periods \(t\).
\item
  \(\psi_t\) represents unique coefficients for the interaction term
  across time periods \(t\).
\item
  \(X_{it}\) is a vector of control variables for category \(i\) at time
  \(t\).
\item
  \(\epsilon_{it}\) is the error term.
\end{itemize}

\section{\texorpdfstring{\textbf{Limitations}}{Limitations}}\label{limitations}

Data only indicates any accidents involving any parked vehicle. The
parked vehicle in question need not be a truck. The parked vehicle may
also be a legally parked truck.~

\section{\texorpdfstring{\textbf{Robustness}}{Robustness}}\label{robustness}

Buses, as a category of large vehicles, may be comparable to trucks in
some contexts, potentially introducing noise into our results. To ensure
robustness, we propose conducting the study both with and without buses
included in the dataset.

Accidents can be categorized as fatalities, injuries, or vehicles
involved, each requiring its own specific dependent variable response.
To address this, we will run our analysis separately for each category,
treating them as controls.

Concerns may arise regarding the time variation of zoning
classifications. However, this concern can be safely excluded, as zoning
regimes or classifications tend to remain relatively stable across
municipalities \citep{mclaughlinLandUseRegulation2012}.

\section{\texorpdfstring{\textbf{Improvements}}{Improvements}}\label{improvements}

There are other data sets found in \citep{NHTSAFileDownloads} or
\citep{FatalityAnalysisReporting} ; . That contain data that could
potentially be more relevant that addresses any limitations in our
design.

Insurance claims dataset are also another dataset worth looking into

\href{https://data.transportation.gov/Trucking-and-Motorcoaches/Motor-Carrier-Crash-Data-/b8e5-isfj/about_data}{Motor
Carrier Crash Data - \textbar{} Department of Transportation - Data
Portal}

\section{Remarks}\label{remarks}

Most research on zoning restrictiveness focuses on housing, yet there is
no reason to believe that a housing restrictiveness index cannot be
adapted for studying industrial zoning restrictiveness.

\section{\texorpdfstring{\textbf{Appendix}}{Appendix}}\label{appendix}

\subsection{\texorpdfstring{\textbf{A. Visualization of
dataset.}}{A. Visualization of dataset.}}\label{sec-a.-visualization-of-dataset.-}

(Present Truck stop parking observatios also available
\href{https://data-usdot.opendata.arcgis.com/datasets/usdot::truck-stop-parking/about}{Truck
Stop Parking \textbar{} Geospatial at the Bureau of Transportation
Statistics})

{[} \citet{coWilliamClintCResearchProposalTrucks2024}{]}

\includegraphics{images/unnamed.png}

\subsection{B. Map of Zoning
Categories}\label{sec-b.-map-of-zoning-categories}

\includegraphics{images/unnamed (1).png}


  \bibliography{bibliography.bib}


\end{document}
